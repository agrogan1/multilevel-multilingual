% Options for packages loaded elsewhere
\PassOptionsToPackage{unicode}{hyperref}
\PassOptionsToPackage{hyphens}{url}
\PassOptionsToPackage{dvipsnames,svgnames,x11names}{xcolor}
%
\documentclass[
  letterpaper,
  DIV=11,
  numbers=noendperiod]{scrreprt}

\usepackage{amsmath,amssymb}
\usepackage{iftex}
\ifPDFTeX
  \usepackage[T1]{fontenc}
  \usepackage[utf8]{inputenc}
  \usepackage{textcomp} % provide euro and other symbols
\else % if luatex or xetex
  \usepackage{unicode-math}
  \defaultfontfeatures{Scale=MatchLowercase}
  \defaultfontfeatures[\rmfamily]{Ligatures=TeX,Scale=1}
\fi
\usepackage{lmodern}
\ifPDFTeX\else  
    % xetex/luatex font selection
\fi
% Use upquote if available, for straight quotes in verbatim environments
\IfFileExists{upquote.sty}{\usepackage{upquote}}{}
\IfFileExists{microtype.sty}{% use microtype if available
  \usepackage[]{microtype}
  \UseMicrotypeSet[protrusion]{basicmath} % disable protrusion for tt fonts
}{}
\makeatletter
\@ifundefined{KOMAClassName}{% if non-KOMA class
  \IfFileExists{parskip.sty}{%
    \usepackage{parskip}
  }{% else
    \setlength{\parindent}{0pt}
    \setlength{\parskip}{6pt plus 2pt minus 1pt}}
}{% if KOMA class
  \KOMAoptions{parskip=half}}
\makeatother
\usepackage{xcolor}
\setlength{\emergencystretch}{3em} % prevent overfull lines
\setcounter{secnumdepth}{5}
% Make \paragraph and \subparagraph free-standing
\ifx\paragraph\undefined\else
  \let\oldparagraph\paragraph
  \renewcommand{\paragraph}[1]{\oldparagraph{#1}\mbox{}}
\fi
\ifx\subparagraph\undefined\else
  \let\oldsubparagraph\subparagraph
  \renewcommand{\subparagraph}[1]{\oldsubparagraph{#1}\mbox{}}
\fi

\usepackage{color}
\usepackage{fancyvrb}
\newcommand{\VerbBar}{|}
\newcommand{\VERB}{\Verb[commandchars=\\\{\}]}
\DefineVerbatimEnvironment{Highlighting}{Verbatim}{commandchars=\\\{\}}
% Add ',fontsize=\small' for more characters per line
\usepackage{framed}
\definecolor{shadecolor}{RGB}{241,243,245}
\newenvironment{Shaded}{\begin{snugshade}}{\end{snugshade}}
\newcommand{\AlertTok}[1]{\textcolor[rgb]{0.68,0.00,0.00}{#1}}
\newcommand{\AnnotationTok}[1]{\textcolor[rgb]{0.37,0.37,0.37}{#1}}
\newcommand{\AttributeTok}[1]{\textcolor[rgb]{0.40,0.45,0.13}{#1}}
\newcommand{\BaseNTok}[1]{\textcolor[rgb]{0.68,0.00,0.00}{#1}}
\newcommand{\BuiltInTok}[1]{\textcolor[rgb]{0.00,0.23,0.31}{#1}}
\newcommand{\CharTok}[1]{\textcolor[rgb]{0.13,0.47,0.30}{#1}}
\newcommand{\CommentTok}[1]{\textcolor[rgb]{0.37,0.37,0.37}{#1}}
\newcommand{\CommentVarTok}[1]{\textcolor[rgb]{0.37,0.37,0.37}{\textit{#1}}}
\newcommand{\ConstantTok}[1]{\textcolor[rgb]{0.56,0.35,0.01}{#1}}
\newcommand{\ControlFlowTok}[1]{\textcolor[rgb]{0.00,0.23,0.31}{#1}}
\newcommand{\DataTypeTok}[1]{\textcolor[rgb]{0.68,0.00,0.00}{#1}}
\newcommand{\DecValTok}[1]{\textcolor[rgb]{0.68,0.00,0.00}{#1}}
\newcommand{\DocumentationTok}[1]{\textcolor[rgb]{0.37,0.37,0.37}{\textit{#1}}}
\newcommand{\ErrorTok}[1]{\textcolor[rgb]{0.68,0.00,0.00}{#1}}
\newcommand{\ExtensionTok}[1]{\textcolor[rgb]{0.00,0.23,0.31}{#1}}
\newcommand{\FloatTok}[1]{\textcolor[rgb]{0.68,0.00,0.00}{#1}}
\newcommand{\FunctionTok}[1]{\textcolor[rgb]{0.28,0.35,0.67}{#1}}
\newcommand{\ImportTok}[1]{\textcolor[rgb]{0.00,0.46,0.62}{#1}}
\newcommand{\InformationTok}[1]{\textcolor[rgb]{0.37,0.37,0.37}{#1}}
\newcommand{\KeywordTok}[1]{\textcolor[rgb]{0.00,0.23,0.31}{#1}}
\newcommand{\NormalTok}[1]{\textcolor[rgb]{0.00,0.23,0.31}{#1}}
\newcommand{\OperatorTok}[1]{\textcolor[rgb]{0.37,0.37,0.37}{#1}}
\newcommand{\OtherTok}[1]{\textcolor[rgb]{0.00,0.23,0.31}{#1}}
\newcommand{\PreprocessorTok}[1]{\textcolor[rgb]{0.68,0.00,0.00}{#1}}
\newcommand{\RegionMarkerTok}[1]{\textcolor[rgb]{0.00,0.23,0.31}{#1}}
\newcommand{\SpecialCharTok}[1]{\textcolor[rgb]{0.37,0.37,0.37}{#1}}
\newcommand{\SpecialStringTok}[1]{\textcolor[rgb]{0.13,0.47,0.30}{#1}}
\newcommand{\StringTok}[1]{\textcolor[rgb]{0.13,0.47,0.30}{#1}}
\newcommand{\VariableTok}[1]{\textcolor[rgb]{0.07,0.07,0.07}{#1}}
\newcommand{\VerbatimStringTok}[1]{\textcolor[rgb]{0.13,0.47,0.30}{#1}}
\newcommand{\WarningTok}[1]{\textcolor[rgb]{0.37,0.37,0.37}{\textit{#1}}}

\providecommand{\tightlist}{%
  \setlength{\itemsep}{0pt}\setlength{\parskip}{0pt}}\usepackage{longtable,booktabs,array}
\usepackage{calc} % for calculating minipage widths
% Correct order of tables after \paragraph or \subparagraph
\usepackage{etoolbox}
\makeatletter
\patchcmd\longtable{\par}{\if@noskipsec\mbox{}\fi\par}{}{}
\makeatother
% Allow footnotes in longtable head/foot
\IfFileExists{footnotehyper.sty}{\usepackage{footnotehyper}}{\usepackage{footnote}}
\makesavenoteenv{longtable}
\usepackage{graphicx}
\makeatletter
\def\maxwidth{\ifdim\Gin@nat@width>\linewidth\linewidth\else\Gin@nat@width\fi}
\def\maxheight{\ifdim\Gin@nat@height>\textheight\textheight\else\Gin@nat@height\fi}
\makeatother
% Scale images if necessary, so that they will not overflow the page
% margins by default, and it is still possible to overwrite the defaults
% using explicit options in \includegraphics[width, height, ...]{}
\setkeys{Gin}{width=\maxwidth,height=\maxheight,keepaspectratio}
% Set default figure placement to htbp
\makeatletter
\def\fps@figure{htbp}
\makeatother
% definitions for citeproc citations
\NewDocumentCommand\citeproctext{}{}
\NewDocumentCommand\citeproc{mm}{%
  \begingroup\def\citeproctext{#2}\cite{#1}\endgroup}
\makeatletter
 % allow citations to break across lines
 \let\@cite@ofmt\@firstofone
 % avoid brackets around text for \cite:
 \def\@biblabel#1{}
 \def\@cite#1#2{{#1\if@tempswa , #2\fi}}
\makeatother
\newlength{\cslhangindent}
\setlength{\cslhangindent}{1.5em}
\newlength{\csllabelwidth}
\setlength{\csllabelwidth}{3em}
\newenvironment{CSLReferences}[2] % #1 hanging-indent, #2 entry-spacing
 {\begin{list}{}{%
  \setlength{\itemindent}{0pt}
  \setlength{\leftmargin}{0pt}
  \setlength{\parsep}{0pt}
  % turn on hanging indent if param 1 is 1
  \ifodd #1
   \setlength{\leftmargin}{\cslhangindent}
   \setlength{\itemindent}{-1\cslhangindent}
  \fi
  % set entry spacing
  \setlength{\itemsep}{#2\baselineskip}}}
 {\end{list}}
\usepackage{calc}
\newcommand{\CSLBlock}[1]{\hfill\break\parbox[t]{\linewidth}{\strut\ignorespaces#1\strut}}
\newcommand{\CSLLeftMargin}[1]{\parbox[t]{\csllabelwidth}{\strut#1\strut}}
\newcommand{\CSLRightInline}[1]{\parbox[t]{\linewidth - \csllabelwidth}{\strut#1\strut}}
\newcommand{\CSLIndent}[1]{\hspace{\cslhangindent}#1}

\KOMAoption{captions}{tableheading}
\makeatletter
\@ifpackageloaded{tcolorbox}{}{\usepackage[skins,breakable]{tcolorbox}}
\@ifpackageloaded{fontawesome5}{}{\usepackage{fontawesome5}}
\definecolor{quarto-callout-color}{HTML}{909090}
\definecolor{quarto-callout-note-color}{HTML}{0758E5}
\definecolor{quarto-callout-important-color}{HTML}{CC1914}
\definecolor{quarto-callout-warning-color}{HTML}{EB9113}
\definecolor{quarto-callout-tip-color}{HTML}{00A047}
\definecolor{quarto-callout-caution-color}{HTML}{FC5300}
\definecolor{quarto-callout-color-frame}{HTML}{acacac}
\definecolor{quarto-callout-note-color-frame}{HTML}{4582ec}
\definecolor{quarto-callout-important-color-frame}{HTML}{d9534f}
\definecolor{quarto-callout-warning-color-frame}{HTML}{f0ad4e}
\definecolor{quarto-callout-tip-color-frame}{HTML}{02b875}
\definecolor{quarto-callout-caution-color-frame}{HTML}{fd7e14}
\makeatother
\makeatletter
\@ifpackageloaded{bookmark}{}{\usepackage{bookmark}}
\makeatother
\makeatletter
\@ifpackageloaded{caption}{}{\usepackage{caption}}
\AtBeginDocument{%
\ifdefined\contentsname
  \renewcommand*\contentsname{Table of contents}
\else
  \newcommand\contentsname{Table of contents}
\fi
\ifdefined\listfigurename
  \renewcommand*\listfigurename{List of Figures}
\else
  \newcommand\listfigurename{List of Figures}
\fi
\ifdefined\listtablename
  \renewcommand*\listtablename{List of Tables}
\else
  \newcommand\listtablename{List of Tables}
\fi
\ifdefined\figurename
  \renewcommand*\figurename{Figure}
\else
  \newcommand\figurename{Figure}
\fi
\ifdefined\tablename
  \renewcommand*\tablename{Table}
\else
  \newcommand\tablename{Table}
\fi
}
\@ifpackageloaded{float}{}{\usepackage{float}}
\floatstyle{ruled}
\@ifundefined{c@chapter}{\newfloat{codelisting}{h}{lop}}{\newfloat{codelisting}{h}{lop}[chapter]}
\floatname{codelisting}{Listing}
\newcommand*\listoflistings{\listof{codelisting}{List of Listings}}
\makeatother
\makeatletter
\makeatother
\makeatletter
\@ifpackageloaded{caption}{}{\usepackage{caption}}
\@ifpackageloaded{subcaption}{}{\usepackage{subcaption}}
\makeatother
\ifLuaTeX
  \usepackage{selnolig}  % disable illegal ligatures
\fi
\usepackage{bookmark}

\IfFileExists{xurl.sty}{\usepackage{xurl}}{} % add URL line breaks if available
\urlstyle{same} % disable monospaced font for URLs
\hypersetup{
  pdftitle={Multilevel Multilingual},
  pdfauthor={Andrew Grogan-Kaylor},
  colorlinks=true,
  linkcolor={blue},
  filecolor={Maroon},
  citecolor={Blue},
  urlcolor={Blue},
  pdfcreator={LaTeX via pandoc}}

\title{Multilevel Multilingual}
\usepackage{etoolbox}
\makeatletter
\providecommand{\subtitle}[1]{% add subtitle to \maketitle
  \apptocmd{\@title}{\par {\large #1 \par}}{}{}
}
\makeatother
\subtitle{Multilevel Models in Stata, R and Julia}
\author{Andrew Grogan-Kaylor}
\date{2024-04-17}

\begin{document}
\maketitle

\renewcommand*\contentsname{Table of contents}
{
\hypersetup{linkcolor=}
\setcounter{tocdepth}{2}
\tableofcontents
}
\listoftables
\bookmarksetup{startatroot}

\chapter{Multilevel Multilingual}\label{multilevel-multilingual}

\begin{quote}
``This curious world which we inhabit is more wonderful than it is
convenient\ldots{}'' (Thoreau, 1975)
\end{quote}

\begin{quote}
``Mathematics is my secret. My secret weakness. I feel like a stubborn,
helpless fool in the middle of a problem. Trapped and crazed. Also,
thrilled.'' (Schanen, 2021)
\end{quote}

\section{Introduction}\label{introduction}

Below, I describe the use of \href{https://www.stata.com/}{Stata}
(StataCorp, 2021), \href{https://www.r-project.org/}{R} (Bates et al.,
2015; R Core Team, 2023), and \href{https://www.julialang.org/}{Julia}
(Bates, 2024; Bezanson et al., 2017) to estimate multilevel models.

All of these software packages can estimate multilevel models. However,
there are substantial differences between the different packages: Stata
is proprietary \emph{for cost} software, which is very well documented
and very intuitive. R is free open source software which is less
intuitive, but there are many excellent resources for learning R. Julia
is newer open source software, and ostensibly much faster than either
Stata or R, which may be an important advantage when running multilevel
models with very large data sets. At this point in time, both Stata and
R feel much more \emph{stable} than Julia which is still evolving
software.

\begin{longtable}[]{@{}
  >{\raggedright\arraybackslash}p{(\columnwidth - 4\tabcolsep) * \real{0.1667}}
  >{\raggedright\arraybackslash}p{(\columnwidth - 4\tabcolsep) * \real{0.1528}}
  >{\raggedright\arraybackslash}p{(\columnwidth - 4\tabcolsep) * \real{0.5417}}@{}}
\caption{Software for Multilevel
Modeling}\label{tbl-software}\tabularnewline
\toprule\noalign{}
\begin{minipage}[b]{\linewidth}\raggedright
Software
\end{minipage} & \begin{minipage}[b]{\linewidth}\raggedright
Cost
\end{minipage} & \begin{minipage}[b]{\linewidth}\raggedright
Ease of Use
\end{minipage} \\
\midrule\noalign{}
\endfirsthead
\toprule\noalign{}
\begin{minipage}[b]{\linewidth}\raggedright
Software
\end{minipage} & \begin{minipage}[b]{\linewidth}\raggedright
Cost
\end{minipage} & \begin{minipage}[b]{\linewidth}\raggedright
Ease of Use
\end{minipage} \\
\midrule\noalign{}
\endhead
\bottomrule\noalign{}
\endlastfoot
Stata & some cost & learning curve, but intuitive for both multilevel
modeling and graphing. \\
R & free & learning curve: intuitive for multilevel modeling; but
steeper learning curve for graphing (\texttt{ggplot}). \\
Julia & free & steep learning curve in general: steep learning curve for
multilevel modeling; and very steep learning curve for graphing.
Graphics libraries are very much under development and in flux. \\
\end{longtable}

\begin{tcolorbox}[enhanced jigsaw, toprule=.15mm, colback=white, colbacktitle=quarto-callout-tip-color!10!white, coltitle=black, opacityback=0, left=2mm, bottomtitle=1mm, toptitle=1mm, rightrule=.15mm, colframe=quarto-callout-tip-color-frame, arc=.35mm, titlerule=0mm, opacitybacktitle=0.6, bottomrule=.15mm, leftrule=.75mm, breakable, title=\textcolor{quarto-callout-tip-color}{\faLightbulb}\hspace{0.5em}{Results Will Vary Somewhat}]

Estimating multilevel models is a complex endeavor. The software details
of how this is accomplished are beyond the purview of this book. Suffice
it to say that across different software packages there will be
differences in estimation routines, resulting in some numerical
differences in the results provided by different software packages.
Substantively speaking, however, results should agree across software.

\end{tcolorbox}

\begin{tcolorbox}[enhanced jigsaw, toprule=.15mm, colback=white, colbacktitle=quarto-callout-tip-color!10!white, coltitle=black, opacityback=0, left=2mm, bottomtitle=1mm, toptitle=1mm, rightrule=.15mm, colframe=quarto-callout-tip-color-frame, arc=.35mm, titlerule=0mm, opacitybacktitle=0.6, bottomrule=.15mm, leftrule=.75mm, breakable, title=\textcolor{quarto-callout-tip-color}{\faLightbulb}\hspace{0.5em}{Multi-Line Commands}]

Sometimes I have written commands out over multiple lines. I have done
this for especially long commands, but have also sometimes done this
simply for the sake of clarity. The different software packages have
different approaches to multi-line commands.

\begin{enumerate}
\def\labelenumi{\arabic{enumi}.}
\tightlist
\item
  By default, \emph{Stata} ends a command at the end of a line. If you
  are going to write a multi-line command you should use the
  \texttt{///} line continuation characters.
\item
  \emph{R} is the software that most naturally can be written using
  multiple lines, as R commands are usually clearly encased in
  parentheses (\texttt{()}) or continued with \texttt{+} signs.
\item
  Like \emph{Stata}, \emph{Julia} expects commands to end at the end of
  a line. If you are going to write a mult-line command, all commands
  except for the last line should end in a character that clearly
  indicates continuation, like a \texttt{+} sign. An alternative is to
  encase the entire Julia command in an outer set of parentheses
  (\texttt{()}).
\end{enumerate}

\end{tcolorbox}

\begin{tcolorbox}[enhanced jigsaw, toprule=.15mm, colback=white, colbacktitle=quarto-callout-tip-color!10!white, coltitle=black, opacityback=0, left=2mm, bottomtitle=1mm, toptitle=1mm, rightrule=.15mm, colframe=quarto-callout-tip-color-frame, arc=.35mm, titlerule=0mm, opacitybacktitle=0.6, bottomrule=.15mm, leftrule=.75mm, breakable, title=\textcolor{quarto-callout-tip-color}{\faLightbulb}\hspace{0.5em}{Running Statistical Packages in Quarto}]

I used Quarto (\url{https://quarto.org/}) to create this Appendix.
Quarto is a programming and publishing environment that can run multiple
programming languages, including Stata, R and Julia, and that can write
to multiple output formats including HTML, PDF, and MS Word. To run
Stata, I used the \texttt{Statamarkdown} library in R to connect Stata
to Quarto. Quarto has a built in connection to R, and runs R without
issue. To run Julia, I used the \texttt{JuliaCall} library in R to
connect Quarto to Julia.

Of course, each of these programs can be run by itself, if you have them
installed on your computer.

\end{tcolorbox}

\section{The Data}\label{sec-data}

The examples use the \texttt{simulated\_multilevel\_data.dta} file from
\href{https://agrogan1.github.io/multilevel-thinking/simulated-multi-country-data.html}{\emph{Multilevel
Thinking}}. Here is a
\href{https://github.com/agrogan1/multilevel-multilingual/raw/main/simulated_multilevel_data.dta}{direct
link} to download the data.

\begin{longtable}[]{@{}
  >{\centering\arraybackslash}p{(\columnwidth - 12\tabcolsep) * \real{0.1266}}
  >{\centering\arraybackslash}p{(\columnwidth - 12\tabcolsep) * \real{0.0759}}
  >{\centering\arraybackslash}p{(\columnwidth - 12\tabcolsep) * \real{0.1139}}
  >{\centering\arraybackslash}p{(\columnwidth - 12\tabcolsep) * \real{0.0759}}
  >{\centering\arraybackslash}p{(\columnwidth - 12\tabcolsep) * \real{0.1392}}
  >{\centering\arraybackslash}p{(\columnwidth - 12\tabcolsep) * \real{0.1899}}
  >{\centering\arraybackslash}p{(\columnwidth - 12\tabcolsep) * \real{0.2785}}@{}}

\caption{\label{tbl-multilingual1}Sample of Simulated Multilevel Data}

\tabularnewline

\caption{Table continues below}\tabularnewline
\toprule\noalign{}
\begin{minipage}[b]{\linewidth}\centering
country
\end{minipage} & \begin{minipage}[b]{\linewidth}\centering
HDI
\end{minipage} & \begin{minipage}[b]{\linewidth}\centering
family
\end{minipage} & \begin{minipage}[b]{\linewidth}\centering
id
\end{minipage} & \begin{minipage}[b]{\linewidth}\centering
identity
\end{minipage} & \begin{minipage}[b]{\linewidth}\centering
intervention
\end{minipage} & \begin{minipage}[b]{\linewidth}\centering
physical\_punishment
\end{minipage} \\
\midrule\noalign{}
\endfirsthead
\toprule\noalign{}
\begin{minipage}[b]{\linewidth}\centering
country
\end{minipage} & \begin{minipage}[b]{\linewidth}\centering
HDI
\end{minipage} & \begin{minipage}[b]{\linewidth}\centering
family
\end{minipage} & \begin{minipage}[b]{\linewidth}\centering
id
\end{minipage} & \begin{minipage}[b]{\linewidth}\centering
identity
\end{minipage} & \begin{minipage}[b]{\linewidth}\centering
intervention
\end{minipage} & \begin{minipage}[b]{\linewidth}\centering
physical\_punishment
\end{minipage} \\
\midrule\noalign{}
\endhead
\bottomrule\noalign{}
\endlastfoot
1 & 69 & 1 & 1.1 & 2 & 1 & 3 \\
1 & 69 & 2 & 1.2 & 2 & 2 & 2 \\
1 & 69 & 3 & 1.3 & 1 & 2 & 3 \\
1 & 69 & 4 & 1.4 & 2 & 1 & 0 \\
1 & 69 & 5 & 1.5 & 2 & 1 & 4 \\
1 & 69 & 6 & 1.6 & 1 & 2 & 5 \\

\end{longtable}

\begin{longtable}[]{@{}
  >{\centering\arraybackslash}p{(\columnwidth - 2\tabcolsep) * \real{0.1250}}
  >{\centering\arraybackslash}p{(\columnwidth - 2\tabcolsep) * \real{0.1389}}@{}}

\caption{\label{tbl-multilingual1}Sample of Simulated Multilevel Data}

\tabularnewline

\toprule\noalign{}
\begin{minipage}[b]{\linewidth}\centering
warmth
\end{minipage} & \begin{minipage}[b]{\linewidth}\centering
outcome
\end{minipage} \\
\midrule\noalign{}
\endhead
\bottomrule\noalign{}
\endlastfoot
3 & 58.47 \\
1 & 51.1 \\
2 & 53.92 \\
5 & 61.17 \\
4 & 56.05 \\
3 & 50.81 \\

\end{longtable}

\section{An Introduction To Equations and Syntax}\label{sec-syntax}

To explain statistical syntax for each software, I consider the general
case of a multilevel model with dependent variable \texttt{y},
independent variables \texttt{x} and \texttt{z}, clustering variable
\texttt{group}, and a random slope for \texttt{x}. \emph{i} is the index
for the person, while \emph{j} is the index for the \texttt{group}.

\begin{equation}\phantomsection\label{eq-MLMsimple}{y = \beta_0 + \beta_1 x_{ij} + \beta_2 z_{ij} + u_{0j} + u_{1j} \times x_{ij} + e_{ij}}\end{equation}

\subsection{Stata}

In Stata \texttt{mixed}, the syntax for a multilevel model of the form
described in Equation~\ref{eq-MLMsimple} is:

\begin{Shaded}
\begin{Highlighting}[]
\NormalTok{mixed }\FunctionTok{y}\NormalTok{ x || }\FunctionTok{group}\NormalTok{: x}
\end{Highlighting}
\end{Shaded}

\subsection{R}

In R \texttt{lme4}, the general syntax for a multilevel model of the
form described in Equation~\ref{eq-MLMsimple} is:

\begin{Shaded}
\begin{Highlighting}[]
\FunctionTok{library}\NormalTok{(lme4)}

\FunctionTok{lmer}\NormalTok{(y }\SpecialCharTok{\textasciitilde{}}\NormalTok{ x }\SpecialCharTok{+}\NormalTok{ z }\SpecialCharTok{+}\NormalTok{ (}\DecValTok{1} \SpecialCharTok{+}\NormalTok{ x }\SpecialCharTok{||}\NormalTok{ group), }\AttributeTok{data =}\NormalTok{ ...)}
\end{Highlighting}
\end{Shaded}

\subsection{Julia}

In Julia \texttt{MixedModels}, the general syntax for a multilevel model
of the form described in Equation~\ref{eq-MLMsimple} is:

\begin{Shaded}
\begin{Highlighting}[]
\ImportTok{using} \BuiltInTok{MixedModels}

\FunctionTok{fit}\NormalTok{(MixedModel, }\PreprocessorTok{@formula}\NormalTok{(y }\OperatorTok{\textasciitilde{}}\NormalTok{ x }\OperatorTok{+}\NormalTok{ z }\OperatorTok{+}\NormalTok{ (}\FloatTok{1} \OperatorTok{+}\NormalTok{ x }\OperatorTok{|}\NormalTok{ group)), data)}
\end{Highlighting}
\end{Shaded}

\bookmarksetup{startatroot}

\chapter{Descriptive Statistics}\label{descriptive-statistics}

\section{Descriptive Statistics}\label{descriptive-statistics-1}

\subsection{Stata}

\begin{Shaded}
\begin{Highlighting}[]

\KeywordTok{use}\NormalTok{ simulated\_multilevel\_data.dta }\CommentTok{// use data}
\end{Highlighting}
\end{Shaded}

We use \texttt{summarize} for \emph{continuous} variables, and
\texttt{tabulate} for \emph{categorical} variables.

\begin{Shaded}
\begin{Highlighting}[]

\KeywordTok{summarize}\NormalTok{ outcome warmth physical\_punishment HDI}

\KeywordTok{tabulate} \KeywordTok{identity}

\KeywordTok{tabulate}\NormalTok{ intervention}
\end{Highlighting}
\end{Shaded}

\begin{verbatim}
    Variable |        Obs        Mean    Std. dev.       Min        Max
-------------+---------------------------------------------------------
     outcome |      3,000    53.43327    6.530996   30.60798   75.83553
      warmth |      3,000    3.521667    1.888399          0          7
physical_p~t |      3,000    2.478667    1.360942          0          5
         HDI |      3,000    64.76667    17.24562         33         87


hypothetica |
 l identity |
      group |
   variable |      Freq.     Percent        Cum.
------------+-----------------------------------
          1 |      1,507       50.23       50.23
          2 |      1,493       49.77      100.00
------------+-----------------------------------
      Total |      3,000      100.00


   recieved |
interventio |
          n |      Freq.     Percent        Cum.
------------+-----------------------------------
          1 |      1,547       51.57       51.57
          2 |      1,453       48.43      100.00
------------+-----------------------------------
      Total |      3,000      100.00
\end{verbatim}

\subsection{R}

\begin{Shaded}
\begin{Highlighting}[]
\FunctionTok{library}\NormalTok{(haven) }\CommentTok{\# read data in Stata format}

\NormalTok{df }\OtherTok{\textless{}{-}} \FunctionTok{read\_dta}\NormalTok{(}\StringTok{"simulated\_multilevel\_data.dta"}\NormalTok{)}
\end{Highlighting}
\end{Shaded}

R's descriptive statistics functions rely heavily on whether a variable
is a \emph{numeric} variable, or a \emph{factor} variable. Below, I
convert two variables to factors (\texttt{factor}) before using
\texttt{summary}\footnote{\texttt{skimr} is an excellent new alternative
  library for generating descriptive statistics in R.} to generate
descriptive statistics.

\begin{Shaded}
\begin{Highlighting}[]
\NormalTok{df}\SpecialCharTok{$}\NormalTok{country }\OtherTok{\textless{}{-}} \FunctionTok{factor}\NormalTok{(df}\SpecialCharTok{$}\NormalTok{country)}

\NormalTok{df}\SpecialCharTok{$}\NormalTok{identity }\OtherTok{\textless{}{-}} \FunctionTok{factor}\NormalTok{(df}\SpecialCharTok{$}\NormalTok{identity)}

\NormalTok{df}\SpecialCharTok{$}\NormalTok{intervention }\OtherTok{\textless{}{-}} \FunctionTok{factor}\NormalTok{(df}\SpecialCharTok{$}\NormalTok{intervention)}

\FunctionTok{summary}\NormalTok{(df)}
\end{Highlighting}
\end{Shaded}

\begin{verbatim}
    country          HDI            family            id            identity
 1      : 100   Min.   :33.00   Min.   :  1.00   Length:3000        1:1507  
 2      : 100   1st Qu.:53.00   1st Qu.: 25.75   Class :character   2:1493  
 3      : 100   Median :70.00   Median : 50.50   Mode  :character           
 4      : 100   Mean   :64.77   Mean   : 50.50                              
 5      : 100   3rd Qu.:81.00   3rd Qu.: 75.25                              
 6      : 100   Max.   :87.00   Max.   :100.00                              
 (Other):2400                                                               
 intervention physical_punishment     warmth         outcome     
 1:1547       Min.   :0.000       Min.   :0.000   Min.   :30.61  
 2:1453       1st Qu.:2.000       1st Qu.:2.000   1st Qu.:49.02  
              Median :2.000       Median :4.000   Median :53.45  
              Mean   :2.479       Mean   :3.522   Mean   :53.43  
              3rd Qu.:3.000       3rd Qu.:5.000   3rd Qu.:57.86  
              Max.   :5.000       Max.   :7.000   Max.   :75.84  
                                                                 
\end{verbatim}

\subsection{Julia}

\begin{Shaded}
\begin{Highlighting}[]
\ImportTok{using} \BuiltInTok{Tables}\NormalTok{, }\BuiltInTok{MixedModels}\NormalTok{, }\BuiltInTok{MixedModelsExtras}\NormalTok{, }\BuiltInTok{StatFiles}\NormalTok{, }\BuiltInTok{DataFrames}\NormalTok{, }\BuiltInTok{CategoricalArrays}\NormalTok{, }\BuiltInTok{DataFramesMeta}

\NormalTok{df }\OperatorTok{=} \FunctionTok{DataFrame}\NormalTok{(}\FunctionTok{load}\NormalTok{(}\StringTok{"simulated\_multilevel\_data.dta"}\NormalTok{))}
\end{Highlighting}
\end{Shaded}

Similarly to R, Julia relies on the idea of \emph{variable type}. I use
\texttt{transform} to convert the appropriate variables to
\emph{categorical} variables.

\begin{Shaded}
\begin{Highlighting}[]
\PreprocessorTok{@transform}\NormalTok{!(df, }\OperatorTok{:}\NormalTok{country }\OperatorTok{=} \FunctionTok{categorical}\NormalTok{(}\OperatorTok{:}\NormalTok{country))}

\PreprocessorTok{@transform}\NormalTok{!(df, }\OperatorTok{:}\NormalTok{identity }\OperatorTok{=} \FunctionTok{categorical}\NormalTok{(}\OperatorTok{:}\NormalTok{identity))}

\PreprocessorTok{@transform}\NormalTok{!(df, }\OperatorTok{:}\NormalTok{intervention }\OperatorTok{=} \FunctionTok{categorical}\NormalTok{(}\OperatorTok{:}\NormalTok{intervention))}
\end{Highlighting}
\end{Shaded}

\begin{Shaded}
\begin{Highlighting}[]

\FunctionTok{describe}\NormalTok{(df) }\CommentTok{\# descriptive statistics}
\end{Highlighting}
\end{Shaded}

\begin{verbatim}
9×7 DataFrame
 Row │ variable             mean     min     median  max      nmissing  eltype ⋯
     │ Symbol               Union…   Any     Union…  Any      Int64     Union  ⋯
─────┼──────────────────────────────────────────────────────────────────────────
   1 │ country                       1.0             30.0            0  Union{ ⋯
   2 │ HDI                  64.7667  33.0    70.0    87.0            0  Union{
   3 │ family               50.5     1.0     50.5    100.0           0  Union{
   4 │ id                            1.1             9.99            0  Union{
   5 │ identity                      1.0             2.0             0  Union{ ⋯
   6 │ intervention                  1.0             2.0             0  Union{
   7 │ physical_punishment  2.47867  0.0     2.0     5.0             0  Union{
   8 │ warmth               3.52167  0.0     4.0     7.0             0  Union{
   9 │ outcome              53.4333  30.608  53.449  75.8355         0  Union{ ⋯
                                                                1 column omitted
\end{verbatim}

\bookmarksetup{startatroot}

\chapter{Unconditional Model}\label{unconditional-model}

An \emph{unconditional} multilevel model is a model with no independent
variables. One should always run an unconditional model as the first
step of a multilevel model in order to get a sense of the way that
variation is apportioned in the model across the different levels.

\section{The Equation}\label{the-equation}

\begin{equation}\phantomsection\label{eq-MLMunconditional}{\text{outcome}_{ij}= \beta_0 + u_{0j} + e_{ij}}\end{equation}

The Intraclass Correlation Coefficient (ICC) is given by:

\begin{equation}\phantomsection\label{eq-ICCunconditional}{\text{ICC} = \frac{var(u_{0j})}{var(u_{0j}) + var(e_{ij})}}\end{equation}

In a two level multilevel model, the ICC provides a measure of the
amount of variation attributable to Level 2.

\section{Run Models}\label{run-models}

\subsection{Stata}

\begin{Shaded}
\begin{Highlighting}[]

\KeywordTok{use}\NormalTok{ simulated\_multilevel\_data.dta }\CommentTok{// use data}
\end{Highlighting}
\end{Shaded}

\begin{Shaded}
\begin{Highlighting}[]

\NormalTok{mixed outcome || country: }\CommentTok{// unconditional model}
  
\end{Highlighting}
\end{Shaded}

\begin{verbatim}
Performing EM optimization ...

Performing gradient-based optimization: 
Iteration 0:  Log likelihood = -9856.1548  
Iteration 1:  Log likelihood = -9856.1548  

Computing standard errors ...

Mixed-effects ML regression                           Number of obs    = 3,000
Group variable: country                               Number of groups =    30
                                                      Obs per group:
                                                                   min =   100
                                                                   avg = 100.0
                                                                   max =   100
                                                      Wald chi2(0)     =     .
Log likelihood = -9856.1548                           Prob > chi2      =     .

------------------------------------------------------------------------------
     outcome | Coefficient  Std. err.      z    P>|z|     [95% conf. interval]
-------------+----------------------------------------------------------------
       _cons |   53.46757   .3539097   151.08   0.000     52.77392    54.16122
------------------------------------------------------------------------------

------------------------------------------------------------------------------
  Random-effects parameters  |   Estimate   Std. err.     [95% conf. interval]
-----------------------------+------------------------------------------------
country: Identity            |
                  var(_cons) |   3.348734   .9702594      1.897816    5.908906
-----------------------------+------------------------------------------------
               var(Residual) |   40.88284   1.060908       38.8555    43.01597
------------------------------------------------------------------------------
LR test vs. linear model: chibar2(01) = 169.64        Prob >= chibar2 = 0.0000
\end{verbatim}

\begin{Shaded}
\begin{Highlighting}[]
  
\KeywordTok{estat}\NormalTok{ icc }\CommentTok{// ICC}
\end{Highlighting}
\end{Shaded}

\begin{verbatim}
Intraclass correlation

------------------------------------------------------------------------------
                       Level |        ICC   Std. err.     [95% conf. interval]
-----------------------------+------------------------------------------------
                     country |   .0757091   .0203761      .0442419    .1265931
------------------------------------------------------------------------------
\end{verbatim}

\subsection{R}

\begin{Shaded}
\begin{Highlighting}[]
\FunctionTok{library}\NormalTok{(haven)}

\NormalTok{df }\OtherTok{\textless{}{-}} \FunctionTok{read\_dta}\NormalTok{(}\StringTok{"simulated\_multilevel\_data.dta"}\NormalTok{)}
\end{Highlighting}
\end{Shaded}

\begin{Shaded}
\begin{Highlighting}[]
\FunctionTok{library}\NormalTok{(lme4) }\CommentTok{\# estimate multilevel models}

\NormalTok{fit0 }\OtherTok{\textless{}{-}} \FunctionTok{lmer}\NormalTok{(outcome }\SpecialCharTok{\textasciitilde{}}\NormalTok{ (}\DecValTok{1} \SpecialCharTok{|}\NormalTok{ country),}
             \AttributeTok{data =}\NormalTok{ df) }\CommentTok{\# unconditional model}

\FunctionTok{summary}\NormalTok{(fit0)}
\end{Highlighting}
\end{Shaded}

\begin{verbatim}
Linear mixed model fit by REML ['lmerMod']
Formula: outcome ~ (1 | country)
   Data: df

REML criterion at convergence: 19712.5

Scaled residuals: 
     Min       1Q   Median       3Q      Max 
-2.97650 -0.68006  0.00936  0.67580  3.03510 

Random effects:
 Groups   Name        Variance Std.Dev.
 country  (Intercept)  3.478   1.865   
 Residual             40.883   6.394   
Number of obs: 3000, groups:  country, 30

Fixed effects:
            Estimate Std. Error t value
(Intercept)    53.47       0.36   148.5
\end{verbatim}

\begin{Shaded}
\begin{Highlighting}[]
\FunctionTok{library}\NormalTok{(performance)}

\NormalTok{performance}\SpecialCharTok{::}\FunctionTok{icc}\NormalTok{(fit0) }\CommentTok{\# ICC}
\end{Highlighting}
\end{Shaded}

\begin{verbatim}
# Intraclass Correlation Coefficient

    Adjusted ICC: 0.078
  Unadjusted ICC: 0.078
\end{verbatim}

\subsection{Julia}

\begin{Shaded}
\begin{Highlighting}[]
\ImportTok{using} \BuiltInTok{Tables}\NormalTok{, }\BuiltInTok{MixedModels}\NormalTok{, }\BuiltInTok{MixedModelsExtras}\NormalTok{, }
\BuiltInTok{StatFiles}\NormalTok{, }\BuiltInTok{DataFrames}\NormalTok{, }\BuiltInTok{CategoricalArrays}\NormalTok{, }\BuiltInTok{DataFramesMeta}

\NormalTok{df }\OperatorTok{=} \FunctionTok{DataFrame}\NormalTok{(}\FunctionTok{load}\NormalTok{(}\StringTok{"simulated\_multilevel\_data.dta"}\NormalTok{))}
\end{Highlighting}
\end{Shaded}

\begin{Shaded}
\begin{Highlighting}[]
\PreprocessorTok{@transform}\NormalTok{!(df, }\OperatorTok{:}\NormalTok{country }\OperatorTok{=} \FunctionTok{categorical}\NormalTok{(}\OperatorTok{:}\NormalTok{country))}
\end{Highlighting}
\end{Shaded}

\begin{Shaded}
\begin{Highlighting}[]

\NormalTok{m0 }\OperatorTok{=} \FunctionTok{fit}\NormalTok{(MixedModel, }
         \PreprocessorTok{@formula}\NormalTok{(outcome }\OperatorTok{\textasciitilde{}}\NormalTok{ (}\FloatTok{1} \OperatorTok{|}\NormalTok{ country)), df) }\CommentTok{\# unconditional model}
\end{Highlighting}
\end{Shaded}

\begin{verbatim}
Linear mixed model fit by maximum likelihood
 outcome ~ 1 + (1 | country)
   logLik   -2 logLik     AIC       AICc        BIC    
 -9856.1548 19712.3097 19718.3097 19718.3177 19736.3288

Variance components:
            Column   Variance Std.Dev.
country  (Intercept)   3.34871 1.82995
Residual              40.88285 6.39397
 Number of obs: 3000; levels of grouping factors: 30

  Fixed-effects parameters:
──────────────────────────────────────────────────
               Coef.  Std. Error       z  Pr(>|z|)
──────────────────────────────────────────────────
(Intercept)  53.4676    0.353908  151.08    <1e-99
──────────────────────────────────────────────────
\end{verbatim}

\begin{Shaded}
\begin{Highlighting}[]

\FunctionTok{icc}\NormalTok{(m0) }\CommentTok{\# ICC}
\end{Highlighting}
\end{Shaded}

\begin{verbatim}
0.07570852291396266
\end{verbatim}

\bookmarksetup{startatroot}

\chapter{Cross Sectional Multilevel
Models}\label{cross-sectional-multilevel-models}

\section{The Equation}\label{the-equation-1}

Recall the general model of Equation~\ref{eq-MLMsimple}, and the syntax
outlined in Section~\ref{sec-syntax}. Below in
Equation~\ref{eq-MLMsubstantive}, we consider a more substantive
example.

\begin{equation}\phantomsection\label{eq-MLMsubstantive}{\text{outcome}_{ij}= \beta_0 + \beta_1 \text{warmth}_{ij} +}\end{equation}

\[\beta_2 \text{physical punishment}_{ij} +\]

\[\beta_3 \text{identity}_{ij} + \beta_4 \text{intervention}_{ij} + \beta_5 \text{HDI}_{ij} +\]

\[u_{0j} + u_{1j} \times \text{warmth}_{ij} + e_{ij}\]

\section{Correlated and Uncorrelated Random
Effects}\label{correlated-and-uncorrelated-random-effects}

Consider the covariance matrix of random effects (e.g.~\(u_{0j}\) and
\(u_{1j}\)). In Equation~\ref{eq-varcovar} the covariances of the random
effects are constrained to be zero.

\begin{equation}\phantomsection\label{eq-varcovar}{\begin{bmatrix}
var(u_{0j}) & 0 \\
0 & var(u_{1j}) 
\end{bmatrix}}\end{equation}

As discussed in the Chapter on multilevel models with cross-sectional
data, however, one can consider a multilevel model in which the random
effects are correlated, as is the case in Equation~\ref{eq-varcovaruns}.

\begin{equation}\phantomsection\label{eq-varcovaruns}{\begin{bmatrix}
var(u_{0j}) & cov(u_{0j}, u_{1j}) \\
cov(u_{0j}, u_{1j}) & var(u_{1j}) 
\end{bmatrix}}\end{equation}

Procedures for estimating models with uncorrelated and correlated random
effects are detailed below (Bates et al., 2015; Bates, 2024; StataCorp,
2021).

\begin{longtable}[]{@{}
  >{\raggedright\arraybackslash}p{(\columnwidth - 4\tabcolsep) * \real{0.1667}}
  >{\raggedright\arraybackslash}p{(\columnwidth - 4\tabcolsep) * \real{0.3056}}
  >{\raggedright\arraybackslash}p{(\columnwidth - 4\tabcolsep) * \real{0.3889}}@{}}
\caption{Correlated and Uncorrelated Random
Effects}\label{tbl-REs}\tabularnewline
\toprule\noalign{}
\begin{minipage}[b]{\linewidth}\raggedright
Software
\end{minipage} & \begin{minipage}[b]{\linewidth}\raggedright
Uncorrelated Random Effects
\end{minipage} & \begin{minipage}[b]{\linewidth}\raggedright
Correlated Random Effects
\end{minipage} \\
\midrule\noalign{}
\endfirsthead
\toprule\noalign{}
\begin{minipage}[b]{\linewidth}\raggedright
Software
\end{minipage} & \begin{minipage}[b]{\linewidth}\raggedright
Uncorrelated Random Effects
\end{minipage} & \begin{minipage}[b]{\linewidth}\raggedright
Correlated Random Effects
\end{minipage} \\
\midrule\noalign{}
\endhead
\bottomrule\noalign{}
\endlastfoot
Stata & default & add option: \texttt{,\ cov(uns)} \\
R & separate random effects from grouping variable with
\texttt{\textbar{}\textbar{}} & separate random effects from grouping
variable with \texttt{\textbar{}} \\
Julia & separate terms for each random effect e.g.
\texttt{(1\ \textbar{}\ group)\ +} \texttt{(0\ +\ x\ \textbar{}\ group)}
& separate random effects from grouping variable with
\texttt{\textbar{}}. \\
\end{longtable}

All models in the examples below are run with \emph{uncorrelated} random
effects, but could just as easily be run with \emph{correlated} random
effects.

\section{Run Models}\label{run-models-1}

\subsection{Stata}

\subsubsection{Get The Data}\label{get-the-data}

\begin{Shaded}
\begin{Highlighting}[]

\KeywordTok{use}\NormalTok{ simulated\_multilevel\_data.dta}
\end{Highlighting}
\end{Shaded}

\subsubsection{Run The Model}\label{run-the-model}

\begin{Shaded}
\begin{Highlighting}[]

\NormalTok{mixed outcome warmth physical\_punishment i.}\KeywordTok{identity}\NormalTok{ i.intervention HDI || country: warmth}
\end{Highlighting}
\end{Shaded}

\begin{verbatim}
Performing EM optimization ...

Performing gradient-based optimization: 
Iteration 0:  Log likelihood = -9626.6279  
Iteration 1:  Log likelihood =  -9626.607  
Iteration 2:  Log likelihood =  -9626.607  

Computing standard errors ...

Mixed-effects ML regression                          Number of obs    =  3,000
Group variable: country                              Number of groups =     30
                                                     Obs per group:
                                                                  min =    100
                                                                  avg =  100.0
                                                                  max =    100
                                                     Wald chi2(5)     = 334.14
Log likelihood =  -9626.607                          Prob > chi2      = 0.0000

-------------------------------------------------------------------------------------
            outcome | Coefficient  Std. err.      z    P>|z|     [95% conf. interval]
--------------------+----------------------------------------------------------------
             warmth |   .8345368   .0637213    13.10   0.000     .7096453    .9594282
physical_punishment |  -.9916657   .0797906   -12.43   0.000    -1.148052   -.8352791
         2.identity |  -.3004767   .2170295    -1.38   0.166    -.7258466    .1248933
     2.intervention |   .6396427   .2174519     2.94   0.003     .2134448    1.065841
                HDI |   -.003228   .0199257    -0.16   0.871    -.0422817    .0358256
              _cons |   52.99991   1.371257    38.65   0.000      50.3123    55.68753
-------------------------------------------------------------------------------------

------------------------------------------------------------------------------
  Random-effects parameters  |   Estimate   Std. err.     [95% conf. interval]
-----------------------------+------------------------------------------------
country: Independent         |
                 var(warmth) |   .0227504   .0257784      .0024689    .2096436
                  var(_cons) |   2.963975   .9737647      1.556777    5.643163
-----------------------------+------------------------------------------------
               var(Residual) |   34.97499   .9097109      33.23668    36.80422
------------------------------------------------------------------------------
LR test vs. linear model: chi2(2) = 205.74                Prob > chi2 = 0.0000

Note: LR test is conservative and provided only for reference.
\end{verbatim}

\subsection{R}

\subsubsection{Get The Data}\label{get-the-data-1}

\begin{Shaded}
\begin{Highlighting}[]
\FunctionTok{library}\NormalTok{(haven)}

\NormalTok{df }\OtherTok{\textless{}{-}} \FunctionTok{read\_dta}\NormalTok{(}\StringTok{"simulated\_multilevel\_data.dta"}\NormalTok{)}
\end{Highlighting}
\end{Shaded}

\subsubsection{Run The Model}\label{run-the-model-1}

\begin{Shaded}
\begin{Highlighting}[]
\NormalTok{fit1 }\OtherTok{\textless{}{-}} \FunctionTok{lmer}\NormalTok{(outcome }\SpecialCharTok{\textasciitilde{}}\NormalTok{ warmth }\SpecialCharTok{+}\NormalTok{ physical\_punishment }\SpecialCharTok{+} 
\NormalTok{               identity }\SpecialCharTok{+}\NormalTok{ intervention }\SpecialCharTok{+}\NormalTok{ HDI }\SpecialCharTok{+}
\NormalTok{               (}\DecValTok{1} \SpecialCharTok{+}\NormalTok{ warmth }\SpecialCharTok{||}\NormalTok{ country),}
             \AttributeTok{data =}\NormalTok{ df)}

\FunctionTok{summary}\NormalTok{(fit1)}
\end{Highlighting}
\end{Shaded}

\begin{verbatim}
Linear mixed model fit by REML ['lmerMod']
Formula: outcome ~ warmth + physical_punishment + identity + intervention +  
    HDI + ((1 | country) + (0 + warmth | country))
   Data: df

REML criterion at convergence: 19268.8

Scaled residuals: 
    Min      1Q  Median      3Q     Max 
-3.9774 -0.6563  0.0187  0.6645  3.6730 

Random effects:
 Groups    Name        Variance Std.Dev.
 country   (Intercept)  3.19056 1.786   
 country.1 warmth       0.02465 0.157   
 Residual              35.01782 5.918   
Number of obs: 3000, groups:  country, 30

Fixed effects:
                     Estimate Std. Error t value
(Intercept)         52.672655   1.479571  35.600
warmth               0.834562   0.064252  12.989
physical_punishment -0.991892   0.079845 -12.423
identity            -0.300350   0.217179  -1.383
intervention         0.639059   0.217603   2.937
HDI                 -0.003395   0.020596  -0.165

Correlation of Fixed Effects:
            (Intr) warmth physc_ idntty intrvn
warmth      -0.121                            
physcl_pnsh -0.145 -0.003                     
identity    -0.213 -0.012 -0.003              
interventin -0.223  0.034  0.022 -0.018       
HDI         -0.902 -0.006  0.009 -0.001  0.000
\end{verbatim}

\subsection{Julia}

\subsubsection{Get The Data}\label{get-the-data-2}

\begin{Shaded}
\begin{Highlighting}[]
\ImportTok{using} \BuiltInTok{Tables}\NormalTok{, }\BuiltInTok{MixedModels}\NormalTok{, }\BuiltInTok{StatFiles}\NormalTok{, }\BuiltInTok{DataFrames}\NormalTok{, }\BuiltInTok{CategoricalArrays}\NormalTok{, }\BuiltInTok{DataFramesMeta}

\NormalTok{df }\OperatorTok{=} \FunctionTok{DataFrame}\NormalTok{(}\FunctionTok{load}\NormalTok{(}\StringTok{"simulated\_multilevel\_data.dta"}\NormalTok{))}
\end{Highlighting}
\end{Shaded}

\subsubsection{Change Country To
Categorical}\label{change-country-to-categorical}

\begin{Shaded}
\begin{Highlighting}[]
\PreprocessorTok{@transform}\NormalTok{!(df, }\OperatorTok{:}\NormalTok{country }\OperatorTok{=} \FunctionTok{categorical}\NormalTok{(}\OperatorTok{:}\NormalTok{country))}
\end{Highlighting}
\end{Shaded}

\subsubsection{Run The Model}\label{run-the-model-2}

\begin{Shaded}
\begin{Highlighting}[]

\NormalTok{m1 }\OperatorTok{=} \FunctionTok{fit}\NormalTok{(MixedModel, }\PreprocessorTok{@formula}\NormalTok{(outcome }\OperatorTok{\textasciitilde{}}\NormalTok{ warmth }\OperatorTok{+}\NormalTok{ physical\_punishment }\OperatorTok{+} 
\NormalTok{               identity }\OperatorTok{+}\NormalTok{ intervention }\OperatorTok{+}\NormalTok{ HDI }\OperatorTok{+}
\NormalTok{               (}\FloatTok{1} \OperatorTok{|}\NormalTok{ country) }\OperatorTok{+}
\NormalTok{               (}\FloatTok{0} \OperatorTok{+}\NormalTok{ warmth }\OperatorTok{|}\NormalTok{ country)), df)}
\end{Highlighting}
\end{Shaded}

\begin{verbatim}
Linear mixed model fit by maximum likelihood
 outcome ~ 1 + warmth + physical_punishment + identity + intervention + HDI + (1 | country) + (0 + warmth | country)
   logLik   -2 logLik     AIC       AICc        BIC    
 -9626.6070 19253.2140 19271.2140 19271.2742 19325.2713

Variance components:
            Column    Variance Std.Dev.   Corr.
country  (Intercept)   2.963849 1.721583
         warmth        0.022756 0.150852   .  
Residual              34.974984 5.913965
 Number of obs: 3000; levels of grouping factors: 30

  Fixed-effects parameters:
─────────────────────────────────────────────────────────────
                          Coef.  Std. Error       z  Pr(>|z|)
─────────────────────────────────────────────────────────────
(Intercept)          52.6608      1.43785     36.62    <1e-99
warmth                0.834537    0.0637228   13.10    <1e-38
physical_punishment  -0.991665    0.0797906  -12.43    <1e-34
identity             -0.300475    0.217029    -1.38    0.1662
intervention          0.639641    0.217452     2.94    0.0033
HDI                  -0.0032286   0.0199255   -0.16    0.8713
─────────────────────────────────────────────────────────────
\end{verbatim}

\bookmarksetup{startatroot}

\chapter{Longitudinal Multilevel
Models}\label{longitudinal-multilevel-models}

\section{The Data}\label{the-data}

The data employed in these examples are a longitudinal extension of the
data described in Section~\ref{sec-data}.

\section{The Equation}\label{the-equation-2}

\begin{equation}\phantomsection\label{eq-MLM-longitudinal}{\text{outcome}_{itj} = \beta_0 + \beta_1 \text{parental warmth}_{itj} + \beta_2 \text{physical punishment}_{itj} + \beta_3 \text{time}_{itj} \ + }\end{equation}

\[\beta_4 \text{identity}_{itj} + \beta_5 \text{intervention}_{itj} + \beta_6 \text{HDI}_{itj} +\]

\[u_{0j} + u_{1j} \times \text{parental warmth}_{itj} \ + \]

\[v_{0i} + v_{1i} \times \text{time}_{itj} + e_{itj}\]

\section{Run Models}\label{run-models-2}

\subsection{Stata}

\subsubsection{Get The Data}\label{get-the-data-3}

\begin{Shaded}
\begin{Highlighting}[]

\KeywordTok{use}\NormalTok{ simulated\_multilevel\_longitudinal\_data.dta}
\end{Highlighting}
\end{Shaded}

\subsubsection{Run The Model}\label{run-the-model-3}

\paragraph{Main Effects Only}\label{main-effects-only}

\begin{Shaded}
\begin{Highlighting}[]

\NormalTok{mixed outcome t warmth physical\_punishment i.}\KeywordTok{identity}\NormalTok{ i.intervention HDI || country: warmth}
\end{Highlighting}
\end{Shaded}

\begin{verbatim}
Performing EM optimization ...

Performing gradient-based optimization: 
Iteration 0:  Log likelihood = -28739.506  
Iteration 1:  Log likelihood = -28739.506  

Computing standard errors ...

Mixed-effects ML regression                         Number of obs    =   9,000
Group variable: country                             Number of groups =      30
                                                    Obs per group:
                                                                 min =     300
                                                                 avg =   300.0
                                                                 max =     300
                                                    Wald chi2(6)     = 1119.81
Log likelihood = -28739.506                         Prob > chi2      =  0.0000

-------------------------------------------------------------------------------------
            outcome | Coefficient  Std. err.      z    P>|z|     [95% conf. interval]
--------------------+----------------------------------------------------------------
                  t |   .9443446   .0756408    12.48   0.000     .7960914    1.092598
             warmth |   .9123903   .0430042    21.22   0.000     .8281035     .996677
physical_punishment |  -.9881587   .0451732   -21.87   0.000    -1.076696   -.8996209
         2.identity |  -.1241465   .1242225    -1.00   0.318     -.367618    .1193251
     2.intervention |   .8575839   .1245179     6.89   0.000     .6135332    1.101635
                HDI |  -.0025173   .0191696    -0.13   0.896    -.0400891    .0350544
              _cons |   51.54528   1.304146    39.52   0.000      48.9892    54.10136
-------------------------------------------------------------------------------------

------------------------------------------------------------------------------
  Random-effects parameters  |   Estimate   Std. err.     [95% conf. interval]
-----------------------------+------------------------------------------------
country: Independent         |
                 var(warmth) |   .0229349   .0135353      .0072136    .0729194
                  var(_cons) |     3.0009   .8550708      1.716768    5.245553
-----------------------------+------------------------------------------------
               var(Residual) |   34.31935   .5130963       33.3283    35.33988
------------------------------------------------------------------------------
LR test vs. linear model: chi2(2) = 767.22                Prob > chi2 = 0.0000

Note: LR test is conservative and provided only for reference.
\end{verbatim}

\paragraph{Interactions With Time}\label{interactions-with-time}

\begin{Shaded}
\begin{Highlighting}[]

\NormalTok{mixed outcome c.t\#\#(c.warmth c.physical\_punishment i.}\KeywordTok{identity}\NormalTok{ i.intervention c.HDI) || country: warmth}
\end{Highlighting}
\end{Shaded}

\begin{verbatim}
Performing EM optimization ...

Performing gradient-based optimization: 
Iteration 0:  Log likelihood = -28738.554  
Iteration 1:  Log likelihood = -28738.554  

Computing standard errors ...

Mixed-effects ML regression                         Number of obs    =   9,000
Group variable: country                             Number of groups =      30
                                                    Obs per group:
                                                                 min =     300
                                                                 avg =   300.0
                                                                 max =     300
                                                    Wald chi2(11)    = 1122.75
Log likelihood = -28738.554                         Prob > chi2      =  0.0000

---------------------------------------------------------------------------------------
              outcome | Coefficient  Std. err.      z    P>|z|     [95% conf. interval]
----------------------+----------------------------------------------------------------
                    t |   .7537359   .3719996     2.03   0.043     .0246302    1.482842
               warmth |   .8198365   .0911059     9.00   0.000     .6412723    .9984008
  physical_punishment |  -1.000348   .1198049    -8.35   0.000    -1.235162   -.7655353
           2.identity |  -.2340191   .3271243    -0.72   0.474     -.875171    .4071328
       2.intervention |   .6597456   .3275877     2.01   0.044     .0176856    1.301806
                  HDI |  -.0005531   .0210866    -0.03   0.979     -.041882    .0407757
                      |
         c.t#c.warmth |   .0463746   .0402459     1.15   0.249    -.0325059    .1252551
                      |
                  c.t#|
c.physical_punishment |   .0061255   .0551491     0.11   0.912    -.1019647    .1142157
                      |
         identity#c.t |
                   2  |   .0548965   .1513015     0.36   0.717     -.241649    .3514421
                      |
     intervention#c.t |
                   2  |   .0990704    .151503     0.65   0.513      -.19787    .3960108
                      |
            c.t#c.HDI |  -.0009791   .0043888    -0.22   0.823    -.0095811    .0076229
                      |
                _cons |   51.92503   1.494157    34.75   0.000     48.99654    54.85352
---------------------------------------------------------------------------------------

------------------------------------------------------------------------------
  Random-effects parameters  |   Estimate   Std. err.     [95% conf. interval]
-----------------------------+------------------------------------------------
country: Independent         |
                 var(warmth) |   .0228292   .0135078      .0071588    .0728013
                  var(_cons) |   3.001849   .8552796       1.71738    5.247001
-----------------------------+------------------------------------------------
               var(Residual) |   34.31227   .5129896      33.32141    35.33258
------------------------------------------------------------------------------
LR test vs. linear model: chi2(2) = 767.35                Prob > chi2 = 0.0000

Note: LR test is conservative and provided only for reference.
\end{verbatim}

\subsection{R}

\subsubsection{Get The Data}\label{get-the-data-4}

\begin{Shaded}
\begin{Highlighting}[]
\FunctionTok{library}\NormalTok{(haven)}

\NormalTok{dfL }\OtherTok{\textless{}{-}} \FunctionTok{read\_dta}\NormalTok{(}\StringTok{"simulated\_multilevel\_longitudinal\_data.dta"}\NormalTok{)}
\end{Highlighting}
\end{Shaded}

\subsubsection{Run The Model}\label{run-the-model-4}

\paragraph{Main Effects Only}\label{main-effects-only-1}

\begin{Shaded}
\begin{Highlighting}[]
\NormalTok{fit2A }\OtherTok{\textless{}{-}} \FunctionTok{lmer}\NormalTok{(outcome }\SpecialCharTok{\textasciitilde{}}\NormalTok{ t }\SpecialCharTok{+}\NormalTok{ warmth }\SpecialCharTok{+}\NormalTok{ physical\_punishment }\SpecialCharTok{+} 
\NormalTok{               identity }\SpecialCharTok{+}\NormalTok{ intervention }\SpecialCharTok{+}\NormalTok{ HDI }\SpecialCharTok{+}
\NormalTok{               (}\DecValTok{1} \SpecialCharTok{|}\NormalTok{ country}\SpecialCharTok{/}\NormalTok{id),}
             \AttributeTok{data =}\NormalTok{ dfL)}

\FunctionTok{summary}\NormalTok{(fit2A)}
\end{Highlighting}
\end{Shaded}

\begin{verbatim}
Linear mixed model fit by REML ['lmerMod']
Formula: 
outcome ~ t + warmth + physical_punishment + identity + intervention +  
    HDI + (1 | country/id)
   Data: dfL

REML criterion at convergence: 57022.7

Scaled residuals: 
    Min      1Q  Median      3Q     Max 
-3.6850 -0.6094 -0.0035  0.6133  3.6792 

Random effects:
 Groups     Name        Variance Std.Dev.
 id:country (Intercept)  8.438   2.905   
 country    (Intercept)  3.675   1.917   
 Residual               26.036   5.103   
Number of obs: 9000, groups:  id:country, 3000; country, 30

Fixed effects:
                      Estimate Std. Error t value
(Intercept)         50.6570397  1.4460656  35.031
t                    0.9433806  0.0658755  14.321
warmth               0.9140307  0.0379336  24.096
physical_punishment -1.0087537  0.0497972 -20.257
identity            -0.1319548  0.1517350  -0.870
intervention         0.8591495  0.1520510   5.650
HDI                  0.0007909  0.0207656   0.038

Correlation of Fixed Effects:
            (Intr) t      warmth physc_ idntty intrvn
t           -0.090                                   
warmth      -0.091 -0.002                            
physcl_pnsh -0.091 -0.007 -0.012                     
identity    -0.152  0.000 -0.013 -0.003              
interventin -0.160  0.000  0.039  0.019 -0.018       
HDI         -0.930  0.000 -0.004  0.005  0.000  0.002
\end{verbatim}

\paragraph{Interactions With Time}\label{interactions-with-time-1}

\begin{Shaded}
\begin{Highlighting}[]
\NormalTok{fit2B }\OtherTok{\textless{}{-}} \FunctionTok{lmer}\NormalTok{(outcome }\SpecialCharTok{\textasciitilde{}}\NormalTok{ t }\SpecialCharTok{*}\NormalTok{(warmth }\SpecialCharTok{+}\NormalTok{ physical\_punishment }\SpecialCharTok{+} 
\NormalTok{               identity }\SpecialCharTok{+}\NormalTok{ intervention }\SpecialCharTok{+}\NormalTok{ HDI) }\SpecialCharTok{+}
\NormalTok{               (}\DecValTok{1} \SpecialCharTok{|}\NormalTok{ country}\SpecialCharTok{/}\NormalTok{id),}
             \AttributeTok{data =}\NormalTok{ dfL)}

\FunctionTok{summary}\NormalTok{(fit2B)}
\end{Highlighting}
\end{Shaded}

\begin{verbatim}
Linear mixed model fit by REML ['lmerMod']
Formula: 
outcome ~ t * (warmth + physical_punishment + identity + intervention +  
    HDI) + (1 | country/id)
   Data: dfL

REML criterion at convergence: 57042.8

Scaled residuals: 
    Min      1Q  Median      3Q     Max 
-3.7118 -0.6092 -0.0024  0.6150  3.6779 

Random effects:
 Groups     Name        Variance Std.Dev.
 id:country (Intercept)  8.436   2.905   
 country    (Intercept)  3.675   1.917   
 Residual               26.046   5.104   
Number of obs: 9000, groups:  id:country, 3000; country, 30

Fixed effects:
                        Estimate Std. Error t value
(Intercept)           51.3432052  1.6670196  30.799
t                      0.5994732  0.4199189   1.428
warmth                 0.8170912  0.0805355  10.146
physical_punishment   -1.0097729  0.1113557  -9.068
identity              -0.2446453  0.3041604  -0.804
intervention           0.6604672  0.3046286   2.168
HDI                    0.0026692  0.0221295   0.121
t:warmth               0.0486211  0.0356217   1.365
t:physical_punishment  0.0004964  0.0494590   0.010
t:identity             0.0563140  0.1318043   0.427
t:intervention         0.0995037  0.1319917   0.754
t:HDI                 -0.0009379  0.0038233  -0.245

Correlation of Fixed Effects:
            (Intr) t      warmth physc_ idntty intrvn HDI    t:wrmt t:phy_
t           -0.504                                                        
warmth      -0.170  0.265                                                 
physcl_pnsh -0.180  0.285 -0.005                                          
identity    -0.266  0.397 -0.013 -0.002                                   
interventin -0.279  0.417  0.039  0.019 -0.017                            
HDI         -0.861  0.206 -0.007  0.012 -0.001  0.003                     
t:warmth     0.151 -0.302 -0.882  0.001  0.011 -0.035  0.006              
t:physcl_pn  0.161 -0.319  0.004 -0.894 -0.001 -0.017 -0.010 -0.003       
t:identity   0.230 -0.458  0.011  0.000 -0.867  0.014  0.001 -0.013  0.002
t:intervntn  0.242 -0.481 -0.035 -0.017  0.014 -0.867 -0.003  0.041  0.019
t:HDI        0.301 -0.596  0.015 -0.027  0.002 -0.007 -0.346 -0.016  0.029
            t:dntt t:ntrv
t                        
warmth                   
physcl_pnsh              
identity                 
interventin              
HDI                      
t:warmth                 
t:physcl_pn              
t:identity               
t:intervntn -0.016       
t:HDI       -0.002  0.008
\end{verbatim}

\subsection{Julia}

\subsubsection{Get The Data}\label{get-the-data-5}

\begin{Shaded}
\begin{Highlighting}[]
\ImportTok{using} \BuiltInTok{Tables}\NormalTok{, }\BuiltInTok{MixedModels}\NormalTok{, }\BuiltInTok{StatFiles}\NormalTok{, }\BuiltInTok{DataFrames}\NormalTok{, }\BuiltInTok{CategoricalArrays}\NormalTok{, }\BuiltInTok{DataFramesMeta}

\NormalTok{dfL }\OperatorTok{=} \FunctionTok{DataFrame}\NormalTok{(}\FunctionTok{load}\NormalTok{(}\StringTok{"simulated\_multilevel\_longitudinal\_data.dta"}\NormalTok{))}
\end{Highlighting}
\end{Shaded}

\subsubsection{Run The Model}\label{run-the-model-5}

\paragraph{Change Country To
Categorical}\label{change-country-to-categorical-1}

\begin{Shaded}
\begin{Highlighting}[]
\PreprocessorTok{@transform}\NormalTok{!(dfL, }\OperatorTok{:}\NormalTok{country }\OperatorTok{=} \FunctionTok{categorical}\NormalTok{(}\OperatorTok{:}\NormalTok{country))}
\end{Highlighting}
\end{Shaded}

\paragraph{Main Effects Only}\label{main-effects-only-2}

\begin{Shaded}
\begin{Highlighting}[]

\NormalTok{m2A }\OperatorTok{=} \FunctionTok{fit}\NormalTok{(MixedModel, }\PreprocessorTok{@formula}\NormalTok{(outcome }\OperatorTok{\textasciitilde{}}\NormalTok{ t }\OperatorTok{+}\NormalTok{ warmth }\OperatorTok{+} 
\NormalTok{                                 physical\_punishment }\OperatorTok{+} 
\NormalTok{                                 identity }\OperatorTok{+}\NormalTok{ intervention }\OperatorTok{+} 
\NormalTok{                                 HDI }\OperatorTok{+}
\NormalTok{                                 (}\FloatTok{1} \OperatorTok{|}\NormalTok{ country) }\OperatorTok{+} 
\NormalTok{                                 (}\FloatTok{0} \OperatorTok{+}\NormalTok{ warmth }\OperatorTok{|}\NormalTok{ country) }\OperatorTok{+}
\NormalTok{                                 (}\FloatTok{1} \OperatorTok{|}\NormalTok{ id)), dfL)}
\end{Highlighting}
\end{Shaded}

\begin{verbatim}
Linear mixed model fit by maximum likelihood
 outcome ~ 1 + t + warmth + physical_punishment + identity + intervention + HDI + (1 | country) + (0 + warmth | country) + (1 | id)
    logLik   -2 logLik      AIC         AICc        BIC     
 -28499.6031  56999.2062  57021.2062  57021.2356  57099.3610

Variance components:
            Column    Variance Std.Dev.   Corr.
id       (Intercept)   8.387258 2.896076
country  (Intercept)   3.166920 1.779584
         warmth        0.010761 0.103736   .  
Residual              26.027344 5.101700
 Number of obs: 9000; levels of grouping factors: 3000, 30

  Fixed-effects parameters:
───────────────────────────────────────────────────────────────
                            Coef.  Std. Error       z  Pr(>|z|)
───────────────────────────────────────────────────────────────
(Intercept)          50.7359        1.37201     36.98    <1e-99
t                     0.943864      0.0658716   14.33    <1e-45
warmth                0.913496      0.0423741   21.56    <1e-99
physical_punishment  -1.0079        0.0497622  -20.25    <1e-90
identity             -0.127692      0.151583    -0.84    0.3996
intervention          0.858997      0.151909     5.65    <1e-07
HDI                  -0.000565959   0.0196433   -0.03    0.9770
───────────────────────────────────────────────────────────────
\end{verbatim}

\paragraph{Interactions With Time}\label{interactions-with-time-2}

\begin{Shaded}
\begin{Highlighting}[]

\NormalTok{m2B }\OperatorTok{=} \FunctionTok{fit}\NormalTok{(MixedModel, }\PreprocessorTok{@formula}\NormalTok{(outcome }\OperatorTok{\textasciitilde{}}\NormalTok{ t }\OperatorTok{*}\NormalTok{ (warmth }\OperatorTok{+} 
\NormalTok{                                 physical\_punishment }\OperatorTok{+} 
\NormalTok{                                 identity }\OperatorTok{+}\NormalTok{ intervention }\OperatorTok{+} 
\NormalTok{                                   HDI) }\OperatorTok{+}
\NormalTok{                                 (}\FloatTok{1} \OperatorTok{|}\NormalTok{ country) }\OperatorTok{+}
\NormalTok{                                 (}\FloatTok{0} \OperatorTok{+}\NormalTok{ warmth }\OperatorTok{|}\NormalTok{ country) }\OperatorTok{+}
\NormalTok{                                 (}\FloatTok{1} \OperatorTok{|}\NormalTok{ id)), dfL)}
\end{Highlighting}
\end{Shaded}

\begin{verbatim}
Linear mixed model fit by maximum likelihood
 outcome ~ 1 + t + warmth + physical_punishment + identity + intervention + HDI + t & warmth + t & physical_punishment + t & identity + t & intervention + t & HDI + (1 | country) + (0 + warmth | country) + (1 | id)
    logLik   -2 logLik      AIC         AICc        BIC     
 -28498.3091  56996.6182  57028.6182  57028.6788  57142.2979

Variance components:
            Column    Variance Std.Dev.   Corr.
id       (Intercept)   8.391748 2.896851
country  (Intercept)   3.170040 1.780461
         warmth        0.010609 0.102999   .  
Residual              26.015905 5.100579
 Number of obs: 9000; levels of grouping factors: 3000, 30

  Fixed-effects parameters:
──────────────────────────────────────────────────────────────────
                                Coef.  Std. Error      z  Pr(>|z|)
──────────────────────────────────────────────────────────────────
(Intercept)              51.4143       1.60324     32.07    <1e-99
t                         0.60349      0.419741     1.44    0.1505
warmth                    0.817076     0.0826636    9.88    <1e-22
physical_punishment      -1.00903      0.111293    -9.07    <1e-18
identity                 -0.238714     0.303996    -0.79    0.4323
intervention              0.660761     0.30445      2.17    0.0300
HDI                       0.00136065   0.0210842    0.06    0.9485
t & warmth                0.0483635    0.0356074    1.36    0.1744
t & physical_punishment   0.0005422    0.0494355    0.01    0.9912
t & identity              0.0554384    0.131745     0.42    0.6739
t & intervention          0.0992809    0.131925     0.75    0.4517
t & HDI                  -0.000955067  0.00382162  -0.25    0.8027
──────────────────────────────────────────────────────────────────
\end{verbatim}

\bookmarksetup{startatroot}

\chapter*{References}\label{references}
\addcontentsline{toc}{chapter}{References}

\markboth{References}{References}

\phantomsection\label{refs}
\begin{CSLReferences}{1}{0}
\bibitem[\citeproctext]{ref-MixedModels}
Bates, D. (2024). \emph{{MixedModels.jl Documentation}}.
\url{https://juliastats.org/MixedModels.jl/stable/}

\bibitem[\citeproctext]{ref-JSSv067i01}
Bates, D., Mächler, M., Bolker, B., \& Walker, S. (2015). Fitting linear
mixed-effects models using lme4. \emph{Journal of Statistical Software},
\emph{67}(1), 1--48. \url{https://doi.org/10.18637/jss.v067.i01}

\bibitem[\citeproctext]{ref-JuliaArticle}
Bezanson, J., Edelman, A., Karpinski, S., \& Shah, V. B. (2017). Julia:
A fresh approach to numerical computing. \emph{SIAM Review},
\emph{59}(1), 65--98. \url{https://doi.org/10.1137/141000671}

\bibitem[\citeproctext]{ref-RProgram}
R Core Team. (2023). \emph{R: A language and environment for statistical
computing}. R Foundation for Statistical Computing.
\url{https://www.R-project.org/}

\bibitem[\citeproctext]{ref-Schanen2021}
Schanen, J. (2021). \emph{Math person ({Strogatz Prize} entry)}.
National Museum of Mathematics.

\bibitem[\citeproctext]{ref-StataCorp2021:2}
StataCorp. (2021). \emph{Stata 17 multilevel mixed effects reference
manual}. Stata Press.

\bibitem[\citeproctext]{ref-Thoreau1975}
Thoreau, H. D. (1975). The commercial spirit of modern times {[}1837{]}.
In J. J. Moldenhauer, E. Moser, \& A. C. Kern (Eds.), \emph{Early essays
and miscellanies}. Princeton University Press.

\end{CSLReferences}



\end{document}
